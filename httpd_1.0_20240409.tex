\documentclass[12pt,a4paper]{article}
\usepackage[utf8]{inputenc}
\usepackage{babel}
\usepackage{rest-api}
\makeindex

\title{acos httpd JSON api format}
\author{Jeff Chiang}

\begin{document}

\maketitle

\section{Introduction}
In this document show the JSON API for acos httpd  use

\section{Usage}
%GETCGI

%SETCGI
\begin{apiRoute}{get}{/GetDevName.cgi}{get device name}
	
	\begin{routeParameter}
	
	\routeParamItem{device\_name}{router name}
	\routeParamItem{pre\_mode}{current router mode}
	\end{routeParameter}
	\begin{routeResponse}{application/json}
		\begin{routeResponseItem}{200}{ok}
			\begin{routeResponseItemBody}
{
  "device_name": "IndoorV2-test",
  "pre_mode": "wifi_router"
}
			\end{routeResponseItemBody}
		\end{routeResponseItem}
	\end{routeResponse}
	
\end{apiRoute}
\begin{apiRoute}{post}{/SetDevName.cgi}{device name setting}
	\begin{routeParameter}
	
	\routeParamItem{device\_name}{device name}
	\end{routeParameter}
	\begin{routeRequest}{application/json}
		\begin{routeRequestBody}
{
  "device_name": "IndoorV2-test"
}
		\end{routeRequestBody}
	\end{routeRequest}
	\begin{routeResponse}{application/json}
		\begin{routeResponseItem}{200}{ok}
			\begin{routeResponseItemBody}
{     
	"result:200
}
			\end{routeResponseItemBody}
		\end{routeResponseItem}
	\end{routeResponse}
\end{apiRoute}
\begin{apiRoute}{get}{/GetHotspot.cgi}{get hotspot info}
	
	\begin{routeParameter}
	
	\routeParamItem{hotspot\_3gpplist}{hotspot list}
	\routeParamItem{hotspot\_id}{hotspot id code}
	\routeParamItem{hotspot\_ip}{hotspot ip address}
	\routeParamItem{hotspot\_key}{hotspot password}
	\routeParamItem{hotspot\_ssid}{wlan setting}
	\end{routeParameter}
	\begin{routeResponse}{application/json}
		\begin{routeResponseItem}{200}{ok}
			\begin{routeResponseItemBody}
{
  "hotspot_ssid": "hostpot-test",
  "hotspot_ip": "12.34.56.78",
  "hotspot_id": "12345678",
  "hotspot_key": "password",
  "hotspot_3gpplist": "Unknown"
}
			\end{routeResponseItemBody}
		\end{routeResponseItem}
	\end{routeResponse}
	
\end{apiRoute}
\begin{apiRoute}{post}{/SetHotspot.cgi}{set hotspot config}
	\begin{routeParameter}
	
	\routeParamItem{hotspot\_3gpplist}{hotspot key}
	\routeParamItem{hotspot\_id}{hotspot id}
	\routeParamItem{hotspot\_ip}{setting hotspot ip address}
	\routeParamItem{hotspot\_key}{hotspot key}
	\routeParamItem{hotspot\_ssid}{setting hotspot ssid}
	\end{routeParameter}
	\begin{routeRequest}{application/json}
		\begin{routeRequestBody}
{
  "hotspot_3gpplist": "Unknown",
  "hotspot_id": "12345678",
  "hotspot_ip": "12.34.56.78",
  "hotspot_key": "password",
  "hotspot_ssid": "hostpot-test"
}
		\end{routeRequestBody}
	\end{routeRequest}
	\begin{routeResponse}{application/json}
		\begin{routeResponseItem}{200}{ok}
			\begin{routeResponseItemBody}
{     
	"result:200
}
			\end{routeResponseItemBody}
		\end{routeResponseItem}
	\end{routeResponse}
\end{apiRoute}
\begin{apiRoute}{get}{/GetHotSpotList.cgi}{null}
	
	\begin{routeParameter}
	
	\routeParamItem{count}{TBD}
	\routeParamItem{hspot\_tbl}{TBD}
	\end{routeParameter}
	\begin{routeResponse}{application/json}
		\begin{routeResponseItem}{200}{ok}
			\begin{routeResponseItemBody}
{
  "count": 2,
  "hspot_tbl": [
    {
      "id": 1,
      "mcc": "192.168.1.2",
      "mnc": "dev1",
      "action": "activate"
    },
    {
      "id": 2,
      "mcc": "192.168.1.3",
      "mnc": "dev2",
      "action": "block"
    }
  ]
}
			\end{routeResponseItemBody}
		\end{routeResponseItem}
	\end{routeResponse}
	
\end{apiRoute}
\begin{apiRoute}{post}{/SetHotSpotList.cgi}{null}
	\begin{routeParameter}
	
	\routeParamItem{count}{TBD}
	\routeParamItem{hspot\_tbl}{TBD}
	\end{routeParameter}
	\begin{routeRequest}{application/json}
		\begin{routeRequestBody}
{
  "count": 2,
  "hspot_tbl": [
    {
      "action": "activate",
      "id": 1,
      "mcc": "192.168.1.2",
      "mnc": "dev1"
    },
    {
      "action": "block",
      "id": 2,
      "mcc": "192.168.1.3",
      "mnc": "dev2"
    }
  ]
}
		\end{routeRequestBody}
	\end{routeRequest}
	\begin{routeResponse}{application/json}
		\begin{routeResponseItem}{200}{ok}
			\begin{routeResponseItemBody}
{     
	"result:200
}
			\end{routeResponseItemBody}
		\end{routeResponseItem}
	\end{routeResponse}
\end{apiRoute}
\begin{apiRoute}{get}{/GetIpv6Mode.cgi}{get hotspot info}
	
	\begin{routeParameter}
	
	\routeParamItem{ipv6\_dns}{ipv6 mandory dns}
	\routeParamItem{ipv6\_dns2}{ipv6 secondary dns}
	\routeParamItem{ipv6\_eanble}{enable ipv6 or not}
	\routeParamItem{ipv6\_id}{ipv6 id}
	\routeParamItem{ipv6\_lanAddr}{ipv6 lan ip address}
	\routeParamItem{ipv6\_lanMode}{ipv6 lan ip address}
	\routeParamItem{ipv6\_type}{ipv6 type}
	\routeParamItem{ipv6\_wanMode}{ipv6 wan (dynamic or static)}
	\routeParamItem{ipv6\_wanipaddr}{ipv6 wan ip}
	\end{routeParameter}
	\begin{routeResponse}{application/json}
		\begin{routeResponseItem}{200}{ok}
			\begin{routeResponseItemBody}
{
  "ipv6_dns": "unknown",
  "ipv6_dns2": "unknown",
  "ipv6_eanble": true,
  "ipv6_id": "unknown",
  "ipv6_lanAddr": "unknown",
  "ipv6_lanMode": "dynammic",
  "ipv6_type": "disable",
  "ipv6_wanMode": "dynammic",
  "ipv6_wanipaddr": "unknown"
}
			\end{routeResponseItemBody}
		\end{routeResponseItem}
	\end{routeResponse}
	
\end{apiRoute}
\begin{apiRoute}{post}{/SetIpv6Mode.cgi}{set hotspot config}
	\begin{routeParameter}
	
	\routeParamItem{ipv6\_dns}{ipv6 mandory dns}
	\routeParamItem{ipv6\_dns2}{ipv6 secondary dns}
	\routeParamItem{ipv6\_eanble}{enable ipv6 or not}
	\routeParamItem{ipv6\_id}{ipv6 id}
	\routeParamItem{ipv6\_lanAddr}{ipv6 lan ip address}
	\routeParamItem{ipv6\_lanMode}{ipv6 lan (dynamic or staitc)}
	\routeParamItem{ipv6\_type}{ipv6 type}
	\routeParamItem{ipv6\_wanMode}{ipv6 wan (dynamic or static)}
	\routeParamItem{ipv6\_wanipaddr}{ipv6 wan ip}
	\end{routeParameter}
	\begin{routeRequest}{application/json}
		\begin{routeRequestBody}
{
  "ipv6_dns": "unknown",
  "ipv6_dns2": "unknown",
  "ipv6_eanble": true,
  "ipv6_id": "unknown",
  "ipv6_lanAddr": "unknown",
  "ipv6_lanMode": "dynammic",
  "ipv6_type": "disable",
  "ipv6_wanMode": "dynammic",
  "ipv6_wanipaddr": "unknown"
}
		\end{routeRequestBody}
	\end{routeRequest}
	\begin{routeResponse}{application/json}
		\begin{routeResponseItem}{200}{ok}
			\begin{routeResponseItemBody}
{     
	"result:200
}
			\end{routeResponseItemBody}
		\end{routeResponseItem}
	\end{routeResponse}
\end{apiRoute}
\begin{apiRoute}{get}{/GetLanReserv.cgi}{get hotspot info}
	
	\begin{routeParameter}
	
	\routeParamItem{count}{wlan setting}
	\routeParamItem{lan\_Reservtbl}{wlan setting}
	\end{routeParameter}
	\begin{routeResponse}{application/json}
		\begin{routeResponseItem}{200}{ok}
			\begin{routeResponseItemBody}
{
  "count": 3,
  "lan_Reservtbl": [
    {
      "id": 1,
      "ipaddr": "192.168.1.2",
      "devname": "dev1",
      "mac": "A1:B2:C3:D4:E5:F1",
      "action": "block"
    },
    {
      "id": 2,
      "ipaddr": "192.168.1.3",
      "devname": "dev3",
      "mac": "A1:B2:C3:D4:E5:F3",
      "action": "allow"
    },
    {
      "id": 3,
      "ipaddr": "192.168.1.4",
      "devname": "dev4",
      "mac": "A1:B2:C3:D4:E5:F4",
      "action": "TBD"
    }
  ]
}
			\end{routeResponseItemBody}
		\end{routeResponseItem}
	\end{routeResponse}
	
\end{apiRoute}
\begin{apiRoute}{post}{/SetLanReserv.cgi}{set hotspot config}
	\begin{routeParameter}
	
	\routeParamItem{count}{null}
	\routeParamItem{lan\_Reservtbl}{null}
	\end{routeParameter}
	\begin{routeRequest}{application/json}
		\begin{routeRequestBody}
{
  "count": 3,
  "lan_Reservtbl": [
    {
      "action": "block",
      "devname": "dev1",
      "id": 1,
      "ipaddr": "192.168.1.2",
      "mac": "A1:B2:C3:D4:E5:F1"
    },
    {
      "action": "allow",
      "devname": "dev3",
      "id": 2,
      "ipaddr": "192.168.1.3",
      "mac": "A1:B2:C3:D4:E5:F3"
    },
    {
      "action": "TBD",
      "devname": "dev4",
      "id": 3,
      "ipaddr": "192.168.1.4",
      "mac": "A1:B2:C3:D4:E5:F4"
    }
  ]
}
		\end{routeRequestBody}
	\end{routeRequest}
	\begin{routeResponse}{application/json}
		\begin{routeResponseItem}{200}{ok}
			\begin{routeResponseItemBody}
{     
	"result:200
}
			\end{routeResponseItemBody}
		\end{routeResponseItem}
	\end{routeResponse}
\end{apiRoute}
\begin{apiRoute}{get}{/GetPortForwarding.cgi}{show port fowarding / triggering rule setting and config}
	
	\begin{routeParameter}
	
	\routeParamItem{lan\_ip}{port forwarding translation LAN IP}
	\routeParamItem{pf\_enable}{port forwarding function on/off}
	\routeParamItem{pf\_table}{port forwarding rule table (id,Name,exPort,inPort,inAddr)}
	\routeParamItem{pt\_enable}{port triggering function on / off}
	\routeParamItem{pt\_table}{port triggering rule table (id,Name,exPort,inPort,inAddr)}
	\end{routeParameter}
	\begin{routeResponse}{application/json}
		\begin{routeResponseItem}{200}{ok}
			\begin{routeResponseItemBody}
{
  "lan_ip": "192.168.1.1",
  "pf_enable": true,
  "pf_table": [
    {
      "id": 1,
      "Name": "tes1",
      "exPort": 11,
      "inPort": 111,
      "ipAddr": "192.168.1.1"
    },
    {
      "id": 2,
      "Name": "tes2",
      "exPort": 12,
      "inPort": 112,
      "ipAddr": "192.168.1.2"
    },
    {
      "id": 3,
      "Name": "tes3",
      "exPort": 13,
      "inPort": 113,
      "ipAddr": "192.168.1.3"
    },
    {
      "id": 4,
      "Name": "tes4",
      "exPort": 14,
      "inPort": 114,
      "ipAddr": "192.168.1.4"
    },
    {
      "id": 5,
      "Name": "tes5",
      "exPort": 15,
      "inPort": 115,
      "ipAddr": "192.168.1.5"
    }
  ],
  "pt_enable": true,
  "pt_table": [
    {
      "id": 1,
      "Enable": true,
      "SName": "http",
      "SType": "",
      "InBound": "11",
      "SUser": "192.168.1.1"
    },
    {
      "id": 2,
      "Enable": true,
      "SName": "ftp",
      "SType": "",
      "InBound": "12",
      "SUser": "192.168.1.2"
    },
    {
      "id": 3,
      "Enable": true,
      "SName": "telnet",
      "SType": "",
      "InBound": "13",
      "SUser": "192.168.1.3"
    }
  ]
}
			\end{routeResponseItemBody}
		\end{routeResponseItem}
	\end{routeResponse}
	
\end{apiRoute}
\begin{apiRoute}{post}{/SetPortForwarding.cgi}{setting each port forwardding / triggering rule and setting}
	\begin{routeParameter}
	
	\routeParamItem{pf\_enable}{null}
	\routeParamItem{pf\_table}{null}
	\routeParamItem{pt\_enable}{null}
	\routeParamItem{pt\_table}{null}
	\end{routeParameter}
	\begin{routeRequest}{application/json}
		\begin{routeRequestBody}
{
  "pf_enable": true,
  "pf_table": [
    {
      "Name": "tes1",
      "exPort": 11,
      "id": 1,
      "inPort": 111,
      "ipAddr": "192.168.1.1"
    },
    {
      "Name": "tes2",
      "exPort": 12,
      "id": 2,
      "inPort": 112,
      "ipAddr": "192.168.1.2"
    },
    {
      "Name": "tes3",
      "exPort": 13,
      "id": 3,
      "inPort": 113,
      "ipAddr": "192.168.1.3"
    },
    {
      "Name": "tes4",
      "exPort": 14,
      "id": 4,
      "inPort": 114,
      "ipAddr": "192.168.1.4"
    },
    {
      "Name": "tes5",
      "exPort": 15,
      "id": 5,
      "inPort": 115,
      "ipAddr": "192.168.1.5"
    }
  ],
  "pt_enable": true,
  "pt_table": [
    {
      "Enable": true,
      "InBound": "11",
      "SName": "http",
      "SUser": "192.168.1.1",
      "Stype": 1,
      "id": 1
    },
    {
      "Enable": false,
      "InBound": "12",
      "SName": "ftp",
      "SUser": "192.168.1.2",
      "Stype": 2,
      "id": 2
    },
    {
      "Enable": true,
      "InBound": "13",
      "SName": "telnet",
      "SUser": "192.168.1.3",
      "Stype": 3,
      "id": 3
    }
  ]
}
		\end{routeRequestBody}
	\end{routeRequest}
	\begin{routeResponse}{application/json}
		\begin{routeResponseItem}{200}{ok}
			\begin{routeResponseItemBody}
{     
	"result:200
}
			\end{routeResponseItemBody}
		\end{routeResponseItem}
	\end{routeResponse}
\end{apiRoute}
\begin{apiRoute}{get}{/GetRouterMode.cgi}{get current router mode and wan ip addres info}
	
	\begin{routeParameter}
	
	\routeParamItem{device\_name}{device name len<64}
	\routeParamItem{enable\_ap\_mode}{router mode setting (AP/router)}
	\routeParamItem{ifconfig}{router WAN network info}
	\end{routeParameter}
	\begin{routeResponse}{application/json}
		\begin{routeResponseItem}{200}{ok}
			\begin{routeResponseItemBody}
{
  "device_name": "IndoorV2-test",
  "enable_ap_mode": "0",
  "ifconfig": {
    "ipaddr": "192.168.7.128",
    "netmask": "255.255.255.0",
    "geteway": "192.168.7.151",
    "dns1_pri": ""
  }
}
			\end{routeResponseItemBody}
		\end{routeResponseItem}
	\end{routeResponse}
	
\end{apiRoute}
\begin{apiRoute}{post}{/SetRouterMode.cgi}{set router mode}
	\begin{routeParameter}
	
	\routeParamItem{router\_mode}{setting router mode setting (AP/router)}
	\end{routeParameter}
	\begin{routeRequest}{application/json}
		\begin{routeRequestBody}
{
  "router_mode": "router"
}
		\end{routeRequestBody}
	\end{routeRequest}
	\begin{routeResponse}{application/json}
		\begin{routeResponseItem}{200}{ok}
			\begin{routeResponseItemBody}
{     
	"result:200
}
			\end{routeResponseItemBody}
		\end{routeResponseItem}
	\end{routeResponse}
\end{apiRoute}
\begin{apiRoute}{get}{/GetUpnp.cgi}{get upnp related info and upnp portforwarding table}
	
	\begin{routeParameter}
	
	\routeParamItem{PortMapTable}{upnp port mapping table}
	\routeParamItem{hiddenAdverTime}{upnp advertise time}
	\routeParamItem{hiddenTimeToLive}{upnp time to live}
	\routeParamItem{hiddenTurnUPnPOn}{enable/disable upnp}
	\end{routeParameter}
	\begin{routeResponse}{application/json}
		\begin{routeResponseItem}{200}{ok}
			\begin{routeResponseItemBody}
{
  "hiddenAdverTime": 111,
  "hiddenTimeToLive": 123,
  "hiddenTurnUPnPOn": true
}
			\end{routeResponseItemBody}
		\end{routeResponseItem}
	\end{routeResponse}
	
\end{apiRoute}
\begin{apiRoute}{post}{/SetUpnp.cgi}{set upnp related config}
	\begin{routeParameter}
	
	\routeParamItem{hiddenAdverTime}{setting upnp advertise time}
	\routeParamItem{hiddenTimeToLive}{setting upnp time to live}
	\routeParamItem{hiddenTurnUPnPOn}{setting upnp on / off time}
	\end{routeParameter}
	\begin{routeRequest}{application/json}
		\begin{routeRequestBody}
{
  "hiddenAdverTime": 111,
  "hiddenTimeToLive": 123,
  "hiddenTurnUPnPOn": true
}
		\end{routeRequestBody}
	\end{routeRequest}
	\begin{routeResponse}{application/json}
		\begin{routeResponseItem}{200}{ok}
			\begin{routeResponseItemBody}
{     
	"result:200
}
			\end{routeResponseItemBody}
		\end{routeResponseItem}
	\end{routeResponse}
\end{apiRoute}
\begin{apiRoute}{get}{/GetVer.cgi}{Get firmeare version}
	
	\begin{routeParameter}
	
	\routeParamItem{SOFTWARE\_VERSION}{Main version}
	\routeParamItem{UI\_VERSION}{sub version}
	\end{routeParameter}
	\begin{routeResponse}{application/json}
		\begin{routeResponseItem}{200}{ok}
			\begin{routeResponseItemBody}
{
  "SOFTWARE_VERSION": "V1.0.0.1",
  "UI_VERSION": "1.0.1"
}
			\end{routeResponseItemBody}
		\end{routeResponseItem}
	\end{routeResponse}
	
\end{apiRoute}

\begin{apiRoute}{get}{/GetVlanBridge.cgi}{Get vlan info}
	
	\begin{routeParameter}
	
	\routeParamItem{bridgePort}{enable/disable vlan}
	\routeParamItem{lan\_portcount}{enable/disable vlan}
	\routeParamItem{vlan\_enable}{enable/disable vlan}
	\routeParamItem{vlan\_group}{enable/disable vlan}
	\routeParamItem{vlan\_mode}{enable/disable vlan}
	\routeParamItem{wirebandCount}{enable/disable vlan}
	\end{routeParameter}
	\begin{routeResponse}{application/json}
		\begin{routeResponseItem}{200}{ok}
			\begin{routeResponseItemBody}
{
  "bridgePort": [
    {
      "id": 0,
      "type": "",
      "Enable": false
    },
    {
      "id": 0,
      "type": "",
      "Enable": false
    },
    {
      "id": 0,
      "type": "",
      "Enable": false
    },
    {
      "id": 0,
      "type": "",
      "Enable": false
    },
    {
      "id": 0,
      "type": "",
      "Enable": false
    }
  ],
  "lan_portcount": 4,
  "vlan_enable": true,
  "vlan_group": [
    {
      "Enable": true,
      "Name": "vlan_10",
      "vlan_id": 0,
      "Priority": 0,
      "wieredPort": "1,2",
      "wierelessBnad": "",
      "action": "active"
    },
    {
      "Enable": true,
      "Name": "vlan_100",
      "vlan_id": 0,
      "Priority": 0,
      "wieredPort": "3,4",
      "wierelessBnad": "",
      "action": "active"
    }
  ],
  "vlan_mode": "bridge",
  "wirebandCount": 2
}
			\end{routeResponseItemBody}
		\end{routeResponseItem}
	\end{routeResponse}
	
\end{apiRoute}
\begin{apiRoute}{post}{/SetVlanBridge.cgi}{Set vlan config}
	\begin{routeParameter}
	
	\routeParamItem{bridgePort}{enable/disable vlan}
	\routeParamItem{lan\_portcount}{enable/disable vlan}
	\routeParamItem{vlan\_enable}{enable/disable vlan}
	\routeParamItem{vlan\_group}{enable/disable vlan}
	\routeParamItem{vlan\_mode}{enable/disable vlan}
	\routeParamItem{wirebandCount}{enable/disable vlan}
	\end{routeParameter}
	\begin{routeRequest}{application/json}
		\begin{routeRequestBody}
{
  "bridgePort": [
    {
      "Enable": true,
      "id": 1,
      "type": "wired"
    },
    {
      "Enable": false,
      "id": 2,
      "type": "wired"
    },
    {
      "Enable": true,
      "id": 3,
      "type": "wired"
    },
    {
      "Enable": true,
      "id": 1,
      "type": "wireless"
    },
    {
      "Enable": true,
      "id": 2,
      "type": "wireless"
    }
  ],
  "lan_portcount": 4,
  "vlan_enable": true,
  "vlan_group": [
    {
      "Enable": true,
      "Name": "vlan_10",
      "Priority": 8,
      "action": "active",
      "vlan_id": 10,
      "wieredPort": "1,2",
      "wierelessBnad": "1"
    },
    {
      "Enable": true,
      "Name": "vlan_100",
      "Priority": 7,
      "action": "active",
      "vlan_id": 100,
      "wieredPort": "3,4",
      "wierelessBnad": "2"
    }
  ],
  "vlan_mode": "bridge",
  "wirebandCount": 2
}
		\end{routeRequestBody}
	\end{routeRequest}
	\begin{routeResponse}{application/json}
		\begin{routeResponseItem}{200}{ok}
			\begin{routeResponseItemBody}
{     
	"result:200
}
			\end{routeResponseItemBody}
		\end{routeResponseItem}
	\end{routeResponse}
\end{apiRoute}
\begin{apiRoute}{get}{/GetWanAdv.cgi}{get WAN / firewall advance setting}
	
	\begin{routeParameter}
	
	\routeParamItem{NatInboundFiltering}{setting dmz enable /disable}
	\routeParamItem{disable\_igmp}{setting dmz enable /disable}
	\routeParamItem{disable\_sip}{setting dmz enable /disable}
	\routeParamItem{dmz\_enable}{setting dmz enable /disable}
	\routeParamItem{dmz\_ipaddr}{setting dmz enable /disable}
	\routeParamItem{response\_ping}{setting dmz enable /disable}
	\routeParamItem{wan\_mtu}{setting dmz enable /disable}
	\end{routeParameter}
	\begin{routeResponse}{application/json}
		\begin{routeResponseItem}{200}{ok}
			\begin{routeResponseItemBody}
{
  "dmz_enable": "Enable",
  "dmz_ipaddr": "192.168.1.0",
  "response_ping": "Disable",
  "disable_sip": "fw_sip_enab",
  "NatInboundFiltering": "Disable",
  "disable_igmp": "Disable",
  "wan_mtu": "1450"
}
			\end{routeResponseItemBody}
		\end{routeResponseItem}
	\end{routeResponse}
	
\end{apiRoute}
\begin{apiRoute}{post}{/SetWanAdv.cgi}{set WAN / firewall advance setting}
	\begin{routeParameter}
	
	\routeParamItem{NatInboundFiltering}{setting dmz enable /disable}
	\routeParamItem{disable\_igmp}{setting dmz enable /disable}
	\routeParamItem{disable\_sip}{setting dmz enable /disable}
	\routeParamItem{dmz\_enable}{setting dmz enable /disable}
	\routeParamItem{dmz\_ipaddr}{setting dmz enable /disable}
	\routeParamItem{response\_ping}{setting dmz enable /disable}
	\routeParamItem{wan\_mtu}{setting dmz enable /disable}
	\end{routeParameter}
	\begin{routeRequest}{application/json}
		\begin{routeRequestBody}
{
  "NatInboundFiltering": "Disable",
  "disable_igmp": "Disable",
  "disable_sip": "fw_sip_enab",
  "dmz_enable": "Enable",
  "dmz_ipaddr": "192.168.1.0",
  "response_ping": "Disable",
  "wan_mtu": "1450"
}
		\end{routeRequestBody}
	\end{routeRequest}
	\begin{routeResponse}{application/json}
		\begin{routeResponseItem}{200}{ok}
			\begin{routeResponseItemBody}
{     
	"result:200
}
			\end{routeResponseItemBody}
		\end{routeResponseItem}
	\end{routeResponse}
\end{apiRoute}
\begin{apiRoute}{get}{/GetWifiManagement.cgi}{Get clould wifi management info}
	
	\begin{routeParameter}
	
	\routeParamItem{enable\_payment}{host IP address or resolve name}
	\routeParamItem{mqtt\_host}{host IP address or resolve name}
	\routeParamItem{mqtt\_password}{host IP address or resolve name}
	\routeParamItem{mqtt\_status}{host IP address or resolve name}
	\routeParamItem{mqtt\_status\_msg}{host IP address or resolve name}
	\routeParamItem{mqtt\_username}{host IP address or resolve name}
	\end{routeParameter}
	\begin{routeResponse}{application/json}
		\begin{routeResponseItem}{200}{ok}
			\begin{routeResponseItemBody}
{
  "mqtt_host": "12.34.56.78",
  "mqtt_username": "mqttHost",
  "mqtt_password": "mqttPassword",
  "mqtt_status": 0,
  "mqtt_status_msg": "Disconnected",
  "enable_payment": ""
}
			\end{routeResponseItemBody}
		\end{routeResponseItem}
	\end{routeResponse}
	
\end{apiRoute}
\begin{apiRoute}{post}{/SetWifiManagement.cgi}{Set clould wifi management info}
	\begin{routeParameter}
	
	\routeParamItem{mqtt\_host}{host IP address or resolve name}
	\routeParamItem{mqtt\_password}{host IP address or resolve name}
	\routeParamItem{mqtt\_username}{host IP address or resolve name}
	\end{routeParameter}
	\begin{routeRequest}{application/json}
		\begin{routeRequestBody}
{
  "mqtt_host": "12.34.56.78",
  "mqtt_password": "mqttPassword",
  "mqtt_username": "mqttHost"
}
		\end{routeRequestBody}
	\end{routeRequest}
	\begin{routeResponse}{application/json}
		\begin{routeResponseItem}{200}{ok}
			\begin{routeResponseItemBody}
{     
	"result:200
}
			\end{routeResponseItemBody}
		\end{routeResponseItem}
	\end{routeResponse}
\end{apiRoute}
\begin{apiRoute}{get}{/GetWirelessAdv.cgi}{getting wireless advanced setting}
	
	\begin{routeParameter}
	
	\routeParamItem{enable\_atf}{country area code}
	\routeParamItem{enable\_beamforming}{country area code}
	\routeParamItem{fronthaul\_2g}{country area code}
	\routeParamItem{fronthaul\_5g}{country area code}
	\routeParamItem{init\_preamble\_2g}{country area code}
	\routeParamItem{init\_preamble\_5g}{country area code}
	\routeParamItem{performance\_boost}{country area code}
	\routeParamItem{rts\_2g}{country area code}
	\routeParamItem{rts\_5g}{country area code}
	\routeParamItem{sku\_name}{country area code}
	\end{routeParameter}
	\begin{routeResponse}{application/json}
		\begin{routeResponseItem}{200}{ok}
			\begin{routeResponseItemBody}
{
  "enable_atf": false,
  "enable_beamforming": true,
  "fronthaul_2g": true,
  "fronthaul_5g": true,
  "init_preamble_2g": "short",
  "init_preamble_5g": "long",
  "performance_boost": false,
  "rts_2g": 1234,
  "rts_5g": 567,
  "sku_name": ""
}
			\end{routeResponseItemBody}
		\end{routeResponseItem}
	\end{routeResponse}
	
\end{apiRoute}
\begin{apiRoute}{post}{/SetWirelessAdv.cgi}{setting wireless advanced setting}
	\begin{routeParameter}
	
	\routeParamItem{enable\_atf}{country area code}
	\routeParamItem{enable\_beamforming}{country area code}
	\routeParamItem{fronthaul\_2g}{country area code}
	\routeParamItem{fronthaul\_5g}{country area code}
	\routeParamItem{init\_preamble\_2g}{country area code}
	\routeParamItem{init\_preamble\_5g}{country area code}
	\routeParamItem{performance\_boost}{country area code}
	\routeParamItem{rts\_2g}{country area code}
	\routeParamItem{rts\_5g}{country area code}
	\routeParamItem{sku\_name}{country area code}
	\end{routeParameter}
	\begin{routeRequest}{application/json}
		\begin{routeRequestBody}
{
  "enable_atf": false,
  "enable_beamforming": true,
  "fronthaul_2g": true,
  "fronthaul_5g": true,
  "init_preamble_2g": "short",
  "init_preamble_5g": "long",
  "performance_boost": false,
  "rts_2g": 1234,
  "rts_5g": 567,
  "sku_name": "WW"
}
		\end{routeRequestBody}
	\end{routeRequest}
	\begin{routeResponse}{application/json}
		\begin{routeResponseItem}{200}{ok}
			\begin{routeResponseItemBody}
{     
	"result:200
}
			\end{routeResponseItemBody}
		\end{routeResponseItem}
	\end{routeResponse}
\end{apiRoute}
\end{document}
